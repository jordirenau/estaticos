\documentclass[11pt,a4paper,twoside,onecolumn,notitlepage,final]{scrartcl}
\usepackage[spanish]{babel}
\usepackage[utf8x]{inputenc}

\usepackage[english]{babel}

\usepackage{ucs}
\usepackage{amsmath}
\usepackage{amsfonts}
\usepackage{amssymb}
\usepackage[left=2.00cm, right=2.00cm, top=2.00cm, bottom=2.00cm]{geometry}
\usepackage{abstract}

\usepackage[colorlinks=true]{hyperref}


\usepackage{siunitx}
\sisetup{detect-all}


%% Encabezados y pié de página
\usepackage{fancyhdr}
\pagestyle{fancy}

\fancyhf{}
%Encabezado
\lhead{\ldots el titulo\ldots{}}
\rhead{\href{mailto:jordi.renau@uchceu.es}{jordi.renau@uchceu.es}}
%Pie de página
\usepackage{lastpage}
\lfoot{Universidad CEU Cardenal Herrera}
\rfoot{Pág. \thepage/\pageref*{LastPage}}



% Este trozo de código adapta las imágenes a los tamaños de impresión que tocan. No preguntes más.
\usepackage{graphicx}
\makeatletter
\def\maxwidth{\ifdim\Gin@nat@width>\linewidth\linewidth\else\Gin@nat@width\fi}
\def\maxheight{\ifdim\Gin@nat@height>\textheight\textheight\else\Gin@nat@height\fi}
\makeatother
% Scale images if necessary, so that they will not overflow the page
% margins by default, and it is still possible to overwrite the defaults
% using explicit options in \includegraphics[width, height, ...]{}
\setkeys{Gin}{width=\maxwidth,height=\maxheight,keepaspectratio}
% Set default figure placement to htbp
\makeatletter
\def\fps@figure{htbp}
\makeatother





\newcommand{\mititulo}{ Mi tutulo aquí }
\newcommand{\misubtitulo}{ Algo por abajo en mejor tamaño }
%\author{El Jordi}
%\subtitle{Algo por abajo en mejor tamaño}
\date{\today}











\begin{document}
	\renewcommand{\tablename}{Tabla}
	
			\begin{flushright}
			\thispagestyle{empty}
	\large
	\textit{Autor}: Jordi Renau (\href{mailto:jordi.renau@uchceu.es}{jordi.renau@uchceu.es}) \\
	Universidad CEU Cardenal Herrera %\\[0.5cm]

	\Huge 
	\mititulo \\[0.5cm]
	\huge
	\misubtitulo \\[0.5cm]
	\small \textit{ \today }
\end{flushright}
		~

\hrule




\hypertarget{esta-es-una-primera-secciuxf3n-del-documento}{%
\section{Esta es una primera sección del
documento}\label{esta-es-una-primera-secciuxf3n-del-documento}}

Este es un fichero mínimo de prueba de templates latex

Ponemos un enlace como latex \href{www.uchceu.es}{así} y también como
\href{www.uchceu.es}{tex a tope} y los construye exactamente igual.

\hypertarget{esta-es-una-segunda-secciuxf3n-donde-tenemos-datos-y-figuras}{%
\section{Esta es una segunda sección donde tenemos datos y
figuras}\label{esta-es-una-segunda-secciuxf3n-donde-tenemos-datos-y-figuras}}

Una figura como esta que referencia como figura \ref{fig:internet}

\begin{figure}
\hypertarget{fig:internet}{%
\centering
\includegraphics[width=0.6\textwidth,height=\textheight]{https://raw.githubusercontent.com/jordirenau/jordirenau.github.io/main/docs/_projects/2018-01-01-dovelar_images/IMG_6163.JPG}
\caption{Viene de internet}\label{fig:internet}
}
\end{figure}

Aquí pongo el logo de la eset a ver si me deja.

\begin{figure}
\hypertarget{fig:logo}{%
\centering
\includegraphics[width=0.3\textwidth,height=\textheight]{https://raw.githubusercontent.com/jordirenau/estaticos/main/00_fac_eset.png}
\caption{Logo ESET}\label{fig:logo}
}
\end{figure}
\end{document}